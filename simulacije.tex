\documentclass[a4paper,11pt]{book}

% Paketi za nas jezik (optimalni)
\usepackage{cmsrb}
\usepackage[OT2,T1]{fontenc}
\usepackage[serbian]{babel}


\usepackage{lmodern}
\usepackage{hyperref}


\makeatletter
\newenvironment{chapquote}[2][2em]
  {\setlength{\@tempdima}{#1}%
   \def\chapquote@author{#2}%
   \parshape 1 \@tempdima \dimexpr\textwidth-2\@tempdima\relax%
   \itshape}
  {\par\normalfont\hfill--\ \chapquote@author\hspace*{\@tempdima}\par\bigskip}
\makeatother

\title{\Huge \textbf{Modelovanje sistema za uzgoj} \\ \huge Modelovanje i simulacije}
\author{\textsc{Aleksandar Stojanović RN97-2018}}


\begin{document}

\maketitle
\tableofcontents

%\mainmatter
\chapter*{Uvod}

\section*{Uzgoj u zastvorenom prostoru}
Kada je reč o zatvorenim prostorima, u glavnom se misli na kontrolisano okruženje koje ima za cilj da olakša razvoj biljke pa kasnije i samih plodova autonomno uz što manju interakciju čoveka.

Ovo se postiže uz pomoć raznih podsistema koji prate i utiču na okruženje biljke.

\section*{Tradicionalan ili zatvoren uzgoj}
Dok nam tradicionalan pristup uzgoju olakšava logistiku i nudi dosta pogodnije mogućnosti za ekspanzije, zatvoren pristup pruža kompletno kontrolu nad samim okruženjem. Pored toga biljka je kopletno izolovana od negativnih spoljašnjih faktora kao što su:

\begin{itemize}
  \item paraziti,
  \item naglih oscilacija temperature,
  \item kritične količine padavina.
\end{itemize}

\noindent Sama činjenica da je biljka u izolovanom okruženju nam omogućava da bliže pratimo njen razvoj. Ovo posebno dolazi do izražaja kod otkrivanja problema u ranim fazama.


\section*{Svrha rada}
Glavna svrha rada je razvoj jednostavnog i efektivnog sistema za uzgoj koji je jeftin i dovoljno jednostavan za upotrebu.\\

\noindent U ovom radu prezentovaću svoj pristup dizajniranja ovakvog sistema. Detaljno cu analizirati principe rada individualnih podsistema koji sačinjavaju ovu jedinicu i simulirati njihov rad.




%%%%%%%%%%%%%%%%
% NEW CHAPTER! %
%%%%%%%%%%%%%%%%
\chapter{Priprema}

%\begin{chapquote}{Author's name, \textit{Source of this quote}}
%``This is a quote and I don't know who said this.''
%\end{chapquote}

\section{Definisanje problema}
Dizajniranje i izrada ovakvih sistema nije lak proces jer zahteva poznavanje vise ne tako povezanih domena nauke.

Glavna prepreka je limitirana količina prostora koja nam je na raspolaganju. Kada je reč o uzgoju u zatvorenim prostorima podrazumeva se da nam je sam prostor jako važan resurs i potrebno je iskoristiti ga što efikasnije. Tek kada je prostor pravilno iskorišćen možemo započeti optimizaciju ostlaih delova sistema.

Da bismo prostor koristili efektivno bitno je da unapred definišemo neke od funkcionalnosti našeg sistema: 

1. Merenje i regulacija temperature,

2. Merenje vlažnosti zemlje i ambijenta,

3. Regulacija svetlosnog ciklusa,

4. Regulacija brzine ventilatora,

5. Zalivanje biljke,

6. Logovanje \\ 

\noindent Imajuci ove funkcije na umu mozemo odrediti grub plan projekta. U sledećoj tački ćemo da detaljo definisati svaku od ovih funkcija kako bismo formirali tehničku specifikaciju.

\section{Definisanje cilja i plana projekta}
Dakle, naš cilj je izrada autonomne jedinice koja radi bez čovekovog prisustva. Kako bismo to postigli moramo se osloniti na nekakvu upravljačku jedinicu koja ce biti zadužena za kontrolu celokupnog sistema. 

\noindent \\ Budući da ovakav sistem zahteva kontinualan rad sto podrazumeva dugoročno opterećenje, Adruino Uno\footnote{Arduino Uno je open-source rešenje u vidu kontrolera za IoT projekte koji pruža mnoštvo mogućnosti. Vise na: https://www.arduino.cc/en/Guide/Introduction} je idealno rešenje jer nudmi stabilnost pod dugoročnim radom i jednostavnu integraciju senzora.

Kao monitor za feedback sistema koristicemo mali I2C\footnote{I2C protokol služi za serijsku komunikaciju sa mikrokontrolerima. Vise na: https://i2c.info/} OLED ekran velicine 0.11 inča kao dugoročno rešenje dok će serijski port biti primarno korišćen u početnim fazama izrade.

\subsection{Merenje temperature i regulacija}
Merenje temperature je krucijalan korak jer je to jedan od glavnih faktora okruženja. Različite biljke zahtevaju različite uslove poput povećane vlage, stoga neophodno je koristiti adekvatan hardver za nase uslove. 

\subsection{Merenje vlažnosti zemlje i ambijenta}
Praćenjem vlažnosti zemlje nam omogućava da automatizovano zalivamo biljku u zavisnosti od njenih potreba. Za razliku od fiksnih ciklusa zalivanja kod kojih može doći do preteranog navodnjavanja ovde se mehanizam za navodnjavanje aktivira samo kada je to potrebno.

\subsection{Regulacija svetlosnog ciklusa}
Različite biljke u različitim fazama razvoja zahtevaju specifične svetlosne cikluse. Stoga, moramo konfigurisati naš kontroler po parametrima biljke kako bismo joj pružili optimalne uslove.

\subsection{Regulacija rada ventilatora}
Ventilatori nam koriste za razmenu vazduha sa okolinom. Sa druge strane, kako utičemo na njihovu brzinu jedinica ce se brže odnosno sporije hladiti.

\subsection{Zalivanje biljke}
Zalivanje biljke je jedan od elementarnih zahteva koje moramo ispuniti. Neophodno je osmisliti sistem za jednako distribuiranje vode po celoj saksiji kako bi vrednosti očitane sa senzora bile sto tačnije.

\subsection{Logovanje}
Logovanje nam omogućava detaljnu analizu procesa i samog rada naše mašine ako se korektno implementira. Znatno olakšava otkrivanje greške ili kvara, pomaže u rešavanju i služi kao output sistema.

U sledećep poglavlju zalazimo u tehničke detalje sistema 

\section{Tehnički detalji}

\chapter{Izgradnja modela}

\section{Konceptualizacija modela}

\section{Kolekcija podataka}

\section{Prevodženje modela}

\section{Verifikacija}

\section{Validacija}

\chapter{Izvršavanje simulacije}

\section{Dizajn eksperimenta}

\section{Izvršavanje i analiza}

\section{Dodatna izvršavanja}

\chapter{Implementacija}

\section{Dokumentacija i izveštaj}

\section{Implementacija}

\end{document}

%\section*{Acknowledgements}
%\begin{itemize}
%\item A special word of thanks goes to Professor Don Knuth\footnote{\url{http://www-cs-faculty.stanford.edu/~uno/}} (for \TeX{}) and Leslie Lamport\footnote{\url{http://www.lamport.org/}} (for \LaTeX{}).
%\item I'll also like to thank Gummi\footnote{\url{http://gummi.midnightcoding.org/}} developers and LaTeXila\footnote{\url{http://projects.gnome.org/latexila/}} development team for their awesome \LaTeX{} editors.
%\item I'm deeply indebted my parents, colleagues and friends for their support and encouragement.
%\end{itemize}

%%%%%%%%%%%%%%%%%%%%%%%%%%%%%%%%%%%%%%%%%%%%%%%%%%%%%%%
% Sample table                                        %
% Source: www1.maths.leeds.ac.uk/latex/TableHelp1.pdf %
%%%%%%%%%%%%%%%%%%%%%%%%%%%%%%%%%%%%%%%%%%%%%%%%%%%%%%%
%\begin{table}[ht]
 % \caption{Sample table} % title of Table
  %\centering % used for centering table
  %\begin{tabular}{c c c c}
  % centered columns (4 columns)
  %\hline\hline %inserts double horizontal lines
  %S. No. & Column\#1 & Column\#2 & Column\#3 \\ [0.5ex]
  % inserts table
  %heading
  %\hline % inserts single horizontal line
  %1 & 50 & 837 & 970 \\
  %2 & 47 & 877 & 230 \\
  %3 & 31 & 25 & 415 \\
  % & 35 & 144 & 2356 \\
  %5 & 45 & 300 & 556 \\ [1ex] % [1ex] adds vertical space
  %\hline %inserts single line
  %\end{tabular}
  %\label{table:nonlin} % is used to refer this table in the text
  %\end{table}