\documentclass[a4paper,11pt]{book}

% Paketi za nas jezik (optimalni)
\usepackage{cmsrb}
\usepackage[OT2,T1]{fontenc}
\usepackage[serbian]{babel}


\usepackage{lmodern}
\usepackage{hyperref}


\makeatletter
\newenvironment{chapquote}[2][2em]
  {\setlength{\@tempdima}{#1}%
   \def\chapquote@author{#2}%
   \parshape 1 \@tempdima \dimexpr\textwidth-2\@tempdima\relax%
   \itshape}
  {\par\normalfont\hfill--\ \chapquote@author\hspace*{\@tempdima}\par\bigskip}
\makeatother

\title{\Huge \textbf{Modelovanje sistema za uzgoj} \\ \huge Modelovanje i simulacije}
\author{\textsc{Aleksandar Stojanović RN97-2018}}


\begin{document}

\maketitle
\tableofcontents

%\mainmatter
\chapter*{Uvod}

\section*{Uzgoj u zastvorenom prostoru}
Kada je reč o zatvorenim prostorima, u glavnom se misli na kontrolisano okruženje koje ima za cilj da olakša razvoj biljke pa kasnije i samih plodova autonomno uz što manju interakciju čoveka.

Ovo se postiže uz pomoć raznih podsistema koji utiču na okruženje biljke u vidu regulacije temperature, vlažnosti ili čak samog ciklusa osvetljenja.
\section*{Tradicionalan ili zatvoren uzgoj}
Dok nam tradicionalan pristup uzgoju olakšava logistiku i nudi dosta pogodnije mogućnosti za ekspanzije, zatvoren pristup pruža kompletno kontrolu nad samim okruženjem. Pored toga biljka je kopletno izolovana od negativnih spoljašnjih faktora kao što su:

\begin{itemize}
  \item paraziti,
  \item naglih oscilacija temperature,
  \item kritične količine padavina.
\end{itemize}


\section*{Svrha rada}
U ovom radu baviću se jednim sistemom za uzgoj kroz detaljnu analizu procesa kontrole okruzenja i davati svoje predloge za poboljšanje. Kao oslonac koristiću 12 koraka za izradu simulacija profesora 


%%%%%%%%%%%%%%%%
% NEW CHAPTER! %
%%%%%%%%%%%%%%%%
\chapter{Priprema}

%\begin{chapquote}{Author's name, \textit{Source of this quote}}
%``This is a quote and I don't know who said this.''
%\end{chapquote}

\section{Definisanje problema}
Kada je reč o uzgoju u zatvorenim prostorima podrazumeva se da nam je sam prostor jako važan resurs i potrebno je iskoristiti ga što efikasnije. Ovim se postiže bolji protok vazduha što za sobom povlači lakšu kontrolu temperature.  \\

Da bismo prostor koristili efektivno bitno je da unapred definišemo neke od funkcionalnosti našeg sistema. 

Neke od funkcija koje treba podržati:

1. Merenje i regulacija temperature,

2. Merenje vlažnosti zemlje i ambijenta,

3. Regulacija svetlosnog ciklusa,

4. Regulacija brzine ventilatora,

5. Zalivanje biljke,

6. Logovanje \\ 

\noindent Imajuci ove funkcije na umu mozemo odrediti grub plan projekta. U sledećoj tački ćemo da detaljo definisati svaku od ovih funkcija kako bismo formirali tehničku specifikaciju.

\section{Definisanje cilja i plana projekta}
Dakle, naš cilj je izrada autonomne jedinice koja radi bez čovekovog prisustva. Kako bismo to postigli moramo 

\noindent \\ Budući da ovakav sistem zahteva kontinualan rad, Adruino Uno\footnote{Arduino Uno je open-source rešenje u vidu kontrolera za IoT projekte koji pruža mnoštvo mogućnosti. Vise na: https://www.arduino.cc/en/Guide/Introduction} je idealno rešenje jer nudmi stabilnost pod dugoročnim radom i jednostavnu integraciju senzora.

\subsection{Merenje temperature i regulacija}
Merenje temperature je krucijalan korak jer je to jedan od glavnih faktora okruženja. Različite biljke zahtevaju različite uslove poput povećane vlage, stoga neophodno je koristiti adekvatan hardver za nase uslove. 

\subsection{Merenje vlažnosti zemlje i ambijenta}
Praćenjem vlažnosti zemlje nam omogućava da automatizovano zalivamo biljku u zavisnosti od njenih potreba. Za razliku od fiksnih ciklusa zalivanja kod kojih može doći do preteranog navodnjavanja ovde se mehanizam za navodnjavanje aktivira samo kada je to potrebno.

\subsection{Regulacija svetlosnog ciklusa}
Različite biljke u različitim fazama razvoja zahtevaju specifične svetlosne cikluse. Stoga, moramo konfigurisati naš kontroler po parametrima biljke kako bismo joj pružili optimalne uslove.

\subsection{Regulacija rada ventilatora}
Ventilatori nam koriste za razmenu vazduha sa okolinom. Sa druge strane, kako utičemo na njihovu brzinu jedinica ce se brže odnosno sporije hladiti.

\subsection{Zalivanje biljke}
Zalivanje biljke je jedan od elementarnih zahteva koje moramo ispuniti. Neophodno je osmisliti sistem za jednako distribuiranje vode po celoj saksiji kako bi vrednosti očitane sa senzora bile sto tačnije.

\subsection{Logovanje}
Logovanje nam omogućava detaljnu analizu procesa i samog rada naše mašine ako se korektno implementira. Znatno olakšava otkrivanje greške ili kvara, pomaže u rešavanju i služi kao output sistema.


 
Pored ovih funkcionalnosti, dodat je jedan OLED ekran od 0.11'' kako bismo imali direktan feedback. Ovim smo znatno olakšali uvid u ponašanje sistema jer nam nije potrebna direkta konekcija sa računarom preko serial porta.

\chapter{Izgradnja modela}

\section{Konceptualizacija modela}

\section{Kolekcija podataka}

\section{Prevodženje modela}

\section{Verifikacija}

\section{Validacija}

\chapter{Izvršavanje simulacije}

\section{Dizajn eksperimenta}

\section{Izvršavanje i analiza}

\section{Dodatna izvršavanja}

\chapter{Implementacija}

\section{Dokumentacija i izveštaj}

\section{Implementacija}

\end{document}

%\section*{Acknowledgements}
%\begin{itemize}
%\item A special word of thanks goes to Professor Don Knuth\footnote{\url{http://www-cs-faculty.stanford.edu/~uno/}} (for \TeX{}) and Leslie Lamport\footnote{\url{http://www.lamport.org/}} (for \LaTeX{}).
%\item I'll also like to thank Gummi\footnote{\url{http://gummi.midnightcoding.org/}} developers and LaTeXila\footnote{\url{http://projects.gnome.org/latexila/}} development team for their awesome \LaTeX{} editors.
%\item I'm deeply indebted my parents, colleagues and friends for their support and encouragement.
%\end{itemize}

%%%%%%%%%%%%%%%%%%%%%%%%%%%%%%%%%%%%%%%%%%%%%%%%%%%%%%%
% Sample table                                        %
% Source: www1.maths.leeds.ac.uk/latex/TableHelp1.pdf %
%%%%%%%%%%%%%%%%%%%%%%%%%%%%%%%%%%%%%%%%%%%%%%%%%%%%%%%
%\begin{table}[ht]
 % \caption{Sample table} % title of Table
  %\centering % used for centering table
  %\begin{tabular}{c c c c}
  % centered columns (4 columns)
  %\hline\hline %inserts double horizontal lines
  %S. No. & Column\#1 & Column\#2 & Column\#3 \\ [0.5ex]
  % inserts table
  %heading
  %\hline % inserts single horizontal line
  %1 & 50 & 837 & 970 \\
  %2 & 47 & 877 & 230 \\
  %3 & 31 & 25 & 415 \\
  % & 35 & 144 & 2356 \\
  %5 & 45 & 300 & 556 \\ [1ex] % [1ex] adds vertical space
  %\hline %inserts single line
  %\end{tabular}
  %\label{table:nonlin} % is used to refer this table in the text
  %\end{table}